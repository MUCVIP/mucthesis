\documentclass[a4paper,openany]{ctexbook}

% \usepackage{ctex}
\usepackage{amsthm,amsmath,amssymb} % 数学符号
\usepackage{mathrsfs}   % 数学符号
\usepackage{newclude}

\usepackage{graphicx}   % 插入图片
\usepackage{subfigure}
\usepackage{booktabs,multirow}   % 表格工具
\usepackage{listings,xcolor}    % 代码环境,颜色
\usepackage{fontspec}   % 字体设置

\usepackage{geometry}   % 纸张设置

\usepackage{gbt7714}    % GB7714-2015风格的参考文献
\bibliographystyle{gbt7714-numerical}

\usepackage{tikz}
\usepackage{newclude}
\usepackage{makecell}
\usepackage{pgfplots}
\usepackage{rotfloat}
\usepackage[final]{pdfpages}
\pgfplotsset{compat=newest}

\usepackage{hyperref}   % 添加PDF超链接方便跳转

%%%%%%%%%%%%%%%
% 双语摘要
\newcommand{\enabstractname}{Abstract}
\newcommand{\cnabstractname}{摘~要}
\newenvironment{enabstract}{%
  \par\small
  \noindent\mbox{}\hfill{\bfseries \zihao{3} \enabstractname}\hfill\mbox{}\par
  \vskip 32bp}{\par\vskip 2.5ex}
\newenvironment{cnabstract}{%
  \par\small
  \noindent\mbox{}\hfill{\heiti \zihao{3} \cnabstractname}\hfill\mbox{}\par
  \vskip 32bp}{\par\vskip 2.5ex}
%%%%%%%%%%%%%%%%%%%%%%%%%%%

\newcommand{\tocleadinglinesymbol}{$\cdot$}
\newcommand{\tocleadingline}{\titlerule*[0.4pc]{\tocleadinglinesymbol}}
\newcommand{\toctextformat}{\songti \zihao{4} \linespread{1.3}}

\usepackage{titletoc}
\titlecontents{chapter}[4em]
{\toctextformat}
{\contentslabel{4em}}
{\hspace*{-4em}}
{~\tocleadingline \contentspage}

\titlecontents{section}[4em]
{\toctextformat}
{\contentslabel{2em}}
{\hspace*{-4em}}
{~\tocleadingline \contentspage}

\titlecontents{subsection}[8em]
{\toctextformat}
{\contentslabel{3em}}
{\hspace*{-2em}}
{~\tocleadingline \contentspage}

\titlecontents{subsubsection}[6em]
{\toctextformat}
{\contentslabel{2em}}
{\hspace*{-2em}}
{~\tocleadingline \contentspage}

\renewcommand{\contentsname}{\heiti \zihao{3} \vspace{28pt} 目~录}

\geometry{
    left=2.7cm,
    right=2.2cm,
    top=2.5cm,
    bottom=2.5cm
}

\lstset{    % 代码环境设置
    %行号
    numbers=left,
    %背景框
    framexleftmargin=10mm,
    frame=none,
    %背景色
    %backgroundcolor=\color[rgb]{1,1,0.76},
    backgroundcolor=\color[RGB]{245,245,244},
    %样式
    basicstyle=\fontspec{Consolas},
    keywordstyle={
        \fontspec{Consolas Bold}
        \color{blue}
    },
    identifierstyle=\fontspec{Consolas},
    numberstyle=\color[RGB]{0,192,192},
    commentstyle={
        \fontspec{Consolas Italic}
        \color[RGB]{0,96,96}
    },
    stringstyle={
        %\fontspec{Consolas}
        \slshape
        \color[RGB]{128,0,0}
    },
    %显示空格
    showstringspaces=false
}

\ctexset{
    chapter = {
        name={第,章\hspace*{-0.5\ccwd}}, number={\chinese{chapter}},
        nameformat={\heiti\zihao{3}},
        format={\centering\sffamily},
        titleformat={\heiti \zihao {3}},
        numberformat={\heiti \zihao {3}},
        beforeskip={-10pt},afterskip={32pt},
        runin={true}
    },
    section = {
        % name={第,节\hspace*{-0.25\ccwd}},
        % indent={2\ccwd},
        number={\arabic{chapter}.\arabic{section}},
        format={\flushleft \sffamily \heiti \zihao {4} \linespread{1.5}},
        beforeskip={14pt},afterskip={28pt},
        fixskip=true,
    },
    subsection = {
        % name={,、\hspace*{-1\ccwd}},
        % indent={2\ccwd},
        number={\arabic{chapter}.\arabic{section}.\arabic{subsection}},
        format={\flushleft \sffamily \heiti \zihao {-4}},
        beforeskip={28pt},afterskip={28pt},
        fixskip=true,
    },
    paragraph = {
        indent={2\ccwd},
        format={\flushleft \sffamily \songti \zihao{-4} \linespread{1.2}},
        beforeskip={-10pt},afterskip={-10pt},
    },
    subsubsection = {
        indent={2\ccwd},
        format={\flushleft \sffamily \songti \zihao{-4} \linespread{1.2}},
        beforeskip={12pt},afterskip={12pt},
        fixskip=true,
    },
    % subparagraph = {
    %     name={(,)},
    %     number={\arabic{subparagraph}},
    %     indent={2\ccwd},
    %     format={\flushleft \sffamily \songti \zihao{-4}},
    % },
}
\usepackage{setspace}
% \onehalfspacing
\linespread{2}  % 行距设置
\setlength{\parskip}{0em}

%%% 定一了一个给段落编号的宏和计数器
%%% 使用前,需将计数器重置\setcounter{paranum}{0}
%%% 使用时,在段前调用宏\parwithnum
\newcounter{paranum}
\newcommand{\parwithnum}{(\refstepcounter{paranum}\theparanum)}


%%%%%%%%%%%%%
\title{基于中央民族大学学位论文格式要求的\LaTeX模板}
\author{\kaishu{民大某学生}}
\date{\today}
%%%%%%%%%%%%

\begin{document}
    \begin{titlepage}
        % 封面
        % \maketitle
        \includepdf{chapters/titlepage.pdf}
    \end{titlepage}

    %%%%%%%%%%%%%%%%%%%%%%%%%%%%%%%

    \pagestyle{plain}
    \pagenumbering{Roman}

    %%%%%%%%%%%%%%%%%%%%%%%%%%%%%%%

    %%%%%%%%%%%%%%%%%%%%%%%%%%%%%%%

\begin{cnabstract}
    \zihao{4}
    这里是中文摘要,随便写写好毕业,字数要求咱也不知道。
    
    \noindent\textbf{关键字: } 中央民族大学;\LaTeX

\end{cnabstract}\addcontentsline{toc}{chapter}{\cnabstractname}

%%%%%%%%%%%%%%%%%%%%%%%%%%%%%%%

    % \begin{cnabstract}
    %     摘要正文
    %     \par\textbf{关键字: } 关键字1;关键字2;关键字3
    %     %“\par在段首,表示另起一行,“\textbf{}”,花括号内的内容加粗显示
    % \end{cnabstract}\addcontentsline{toc}{chapter}{\cnabstractname}
    
    % \newpage
    %%%%%%%%%%%%%%%%%%%%%%%%%%%%%%%

\begin{enabstract}
    \zihao{4}

    This is English abstract. Translate the Chinese version with Google Translate and paste it here.

    \noindent\textbf{Keywords:} Minzu University of China; \LaTeX

\end{enabstract}\addcontentsline{toc}{chapter}{\enabstractname}

%%%%%%%%%%%%%%%%%%%%%%%%%%%%%%%

    % \begin{enabstract}
    %     English abstract
    % \par\textbf{Keywords:} keyword1; keyword2; keyword3
    % %“\par在段首,表示另起一行,“\textbf{}”,花括号内的内容加粗显示
    % \end{enabstract}\addcontentsline{toc}{chapter}{\enabstractname}

    %%%%%%%%%%%%%%%%%%%%%%%%%%%%%%%

    \tableofcontents

    \renewcommand{\listfigurename}{插图目录}
    \listoffigures

    \renewcommand{\listtablename}{表格目录}
    \listoftables

    %%%%%%%%%%%%%%%%%%%%%%%%%%%%%%%

    \chapter{绪论}
    \pagenumbering{arabic}
    \setcounter{page}{1}
    \include*{chapters/introduction}

    %%%%%%%%%%%%%%%%%%%%%%%%%%%%%%%

    \chapter{数学公式}
    \include*{chapters/math-shit}

    %%%%%%%%%%%%%%%%%%%%%%%%%%%%%%%

    \chapter{图文混排}
    \include*{chapters/figure}

    %%%%%%%%%%%%%%%%%%%%%%%%%%%%%%%

    \chapter{表格}
    \include*{chapters/table}

    %%%%%%%%%%%%%%%%%%%%%%%%%%%%%%%

    \chapter{总结与展望}
    \include*{chapters/conclusion}

    %%%%%%%%%%%%%%%%%%%%%%%%%%%%%%%

    \bibliography{chapters/reference}\addcontentsline{toc}{chapter}{参考文献}



    \chapter*{致谢}\addcontentsline{toc}{chapter}{致谢}
    \include*{chapters/thankyou}

\end{document}
